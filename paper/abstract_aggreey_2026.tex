%%%% kr-instructions.tex -- version 1.3 (11-Jan-2021)

\typeout{KR2025 Instructions for Authors}

% These are the instructions for authors for KR-25.

\documentclass{article}
\pdfpagewidth=8.5in
\pdfpageheight=11in

\usepackage{kr}

% Use the postscript times font!
\usepackage{times}
\usepackage{soul}
\usepackage{url}
\usepackage[hidelinks]{hyperref}
\usepackage[utf8]{inputenc}
\usepackage[small]{caption}
\usepackage{graphicx}
\usepackage{amsmath}
\usepackage{amsthm}
\usepackage{booktabs}
\usepackage{algorithm}
\usepackage{algorithmic}
\urlstyle{same}

\usepackage{lipsum}

% the following package is optional:
%\usepackage{latexsym}

% See https://www.overleaf.com/learn/latex/theorems_and_proofs
% for a nice explanation of how to define new theorems, but keep
% in mind that the amsthm package is already included in this
% template and that you must *not* alter the styling.
\newtheorem{example}{Example}
\newtheorem{theorem}{Theorem}

% Following comment is from ijcai97-submit.tex:
% The preparation of these files was supported by Schlumberger Palo Alto
% Research, AT\&T Bell Laboratories, and Morgan Kaufmann Publishers.
% Shirley Jowell, of Morgan Kaufmann Publishers, and Peter F.
% Patel-Schneider, of AT\&T Bell Laboratories collaborated on their
% preparation.

% These instructions can be modified and used in other conferences as long
% as credit to the authors and supporting agencies is retained, this notice
% is not changed, and further modification or reuse is not restricted.
% Neither Shirley Jowell nor Peter F. Patel-Schneider can be listed as
% contacts for providing assistance without their prior permission.

% To use for other conferences, change references to files and the
% conference appropriate and use other authors, contacts, publishers, and
% organizations.
% Also change the deadline and address for returning papers and the length and
% page charge instructions.
% Put where the files are available in the appropriate places.
%PDF Info Is REQUIRED.
\pdfinfo{
/TemplateVersion (KR.2022.0, KR.2023.0, KR.2024.0, KR.2025.0)
}



\title{ChessLab: \\ A Framework for Measuring Argumentation Methodology Effectiveness }

% Single author syntax
\author{
    Mario Larsen
    \affiliations
    Independent Researcher
    \emails
    uspectacle@gmail.com
}

\begin{document}

\maketitle

\section*{Introduction}

The \emph{wisdom of crowds} hypothesis states that aggregating the judgments of diverse and independent agents can yield decisions superior to those of individuals \cite{surowiecki2004}.
While this effect is well documented in estimation and forecasting tasks, its applicability to complex, sequential decision-making domains such as chess remains an active area of inquiry.

Historical crowd-versus-expert chess matches and large online events \cite{chess2025} suggest that crowds can be competitive, yet these settings are often anecdotal and lack rigorous experimental control.
More recent systematic evidence demonstrated that large crowds of casual players could rival 1900 Elo artificial opponents over hundreds of games \cite{moussaid2025}.
However, research indicates that simple majority voting may not always be optimal \cite{prelec2017}.

Chess engine systems, ranging from search-based algorithms to human-mimetic neural networks \cite{maia2024}, offer adversaries with precisely controllable strength, diversity, and reproducibility.
At the same time, large language models are beginning to reach grandmaster levels, showing signs of an emerging internal model of the game \cite{karvonen2024,carlini2023}.
This motivates the development of a controlled framework in which collective intelligence can be studied without the confounds inherent to human experiments.
We propose \textbf{ChessLab} to fill this gap by enabling large-scale, reproducible experiments on artificial crowds in chess, with explicit control over ensemble composition and aggregation mechanisms.

\section*{Relevance to Argumentation}

This work is intended to serve as a tool for studying voting mechanisms and deliberation strategies between artificial agents in a chess context.
Chess provides an objective evaluation signal (game outcomes and engine-based position scores) that allows the effectiveness of various methodologies to be compared with arbitrary precision.

\section*{Conclusion}

The modular framework supports controlled manipulation of agent diversity, aggregation rules, and opponent strength and provide a range of evaluation and visualization tools.
Results are computed and analysed in parallel, offering the ability to scale experiments and test variables independently.
The source code is available at \url{https://github.com/Uspectacle/chesslab} under the GNU v3 License.

Although our current main contribution lies in providing this infrastructure, we also provide experimental results to compare against previous studies conducted on humans.
An accompanying roadmap outlines a sequence of experiments designed to explore optimized deliberation mechanisms.

\section*{Acknowledgments}

This research is conducted independently by the author while awaiting a PhD opportunity in collaborative writing and collective intelligence.
Due to limited compute capacity, only lower-strength engines were experimented upon.
Nonetheless, this project aims to enable further experimentation and interesting discoveries.



%% The file kr.bst is a bibliography style file for BibTeX 0.99c
\bibliographystyle{kr}
\bibliography{references}

\end{document}

